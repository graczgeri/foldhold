\documentclass[a4paper, 12pt]{article}
\usepackage{graphicx}

\usepackage[utf8]{inputenc}
\usepackage[magyar]{babel}

% a szép matematikai szimbólumokért
\usepackage{amssymb}
\usepackage{amsmath}

% ha táblázatban szeretnénk egyesített sorokat is
\usepackage{multirow}

% horizontal line
\usepackage{hhline}

% eps formátumú ábrák --> pdflatex fordításhoz!!
\usepackage{epsfig}

% egyenletekhez, pl mátrixok írására
\usepackage{array}

% ha betűszíneket is szeretnénk használni
\usepackage{color}

% Margók egyéni beállításai
\usepackage{anysize}
\marginsize{1.64cm}{1.64cm}{1.2cm}{2.4cm} %\left right top bottom

% vakszöveg
\usepackage{lipsum}
\usepackage{blindtext}

% A HIVATKOZÁSOKHOZ HASZNÁLT CSOMAGOK. RÉSZLETESEBBEN LD: google -> latex bibtex
\usepackage[numbers, square, comma, sort&compress]{natbib}
%\usepackage[format=hang,labelsep=period]{caption}

\usepackage[unicode]{hyperref}   % ezzel a hivatkozások linkké válnak
\usepackage{bookmark}
\hypersetup{bookmarksopen={true}}
\hypersetup{bookmarksopenlevel={2}}
\hypersetup{bookmarksnumbered={true}}
\hypersetup{
  colorlinks,%
  citecolor=red,%
  filecolor=black,%                                                                                                                                               
  linkcolor=blue,%
  urlcolor=green
}
%\numberwithin{equation}{section}          % ezekkel tudod beállítani, hogy milyen felbontásig menjen a hivatkozás
%\numberwithin{figure}{subsection}
%\numberwithin{table}{section}          % ha kikommenteled, akkor csak simán számozva lesz.

%%%%%%%%%%%%%%%% Néhány dolog a fancy kinézethez

\frenchspacing
\setlength{\parskip}{2ex}
\setlength{\headsep}{0,4cm}
\setlength{\headheight}{4pt}

% fej- es lábléc
\usepackage{fancyhdr}
\usepackage{fancyref}
\usepackage{fancyvrb}
\pagestyle{fancy}

\renewcommand{\headrulewidth}{0,05pt}
\renewcommand{\footrulewidth}{0pt}


\fancyhf{}
\fancyhead[RE]{{ \nouppercase{\leftmark}} }
\fancyhead[LO]{{ \nouppercase{\leftmark}} }
\cfoot{--~\thepage~--}


%%%%%%%%%%%%%%%%%%%%%%%%%%%


%%%%%%%%%%%%%%%%%%%%%%%%%%%%%%%%%%%%%%%%%%%%%%%%%%%%%%%%%%%%%%%%%%%%%%%%%%%%

\begin{document}

% Címoldalt lehet egyszerűen a \maketitle paranccsal is. Ha kissé részletesebb
% címre van szükség, azt lehet így is, kézzel megadva mindent.
\begin{titlepage}   
\begin{center}
\thispagestyle{empty}  

\vspace*{0.7cm}
\rule{\linewidth}{0.5mm} \\[3mm]
\vspace*{0.7cm}

{\LARGE Kutatómunka információs eszközei}

\vspace*{0.7cm}
\rule{\linewidth}{0.5mm} \\[3mm]
\rule{\linewidth}{0.5mm} \\[3mm]



{\Large Differenciálegyenletek numerikus integrálási módszerei\\}

\vspace*{0.7cm}
\rule{\linewidth}{0.5mm} \\[3mm]
{\footnotesize Írta: Rácz Gergely} \\
{\tiny graczgeri@gmail.com}
  \vspace*{2cm}

\begin{figure}[h!]
\begin{center}
\includegraphics[width=0.5\textwidth]{img/elte}
\end{center}
\end{figure}

\end{center}
\end{titlepage}

\newpage

%%%%%%%%%%%%%%%%%%%%%%%%%%%%%%%%%%%%%%%%%%%%%%%%%%%%%%%%%%%%%%%%%%%%%%%%

\thispagestyle{empty}  
  \vspace*{3cm}
\begin{abstract}
A megírt programok a Föld-Hold rendszer mozgásegyenletét oldják meg, numerikus integrálási módszerekkel. Két módszert, az explicit Euler módszert és az  adaptív lépéshossz-szabályozott negyedrendű Runge-Kutta-módszert hasonlítottam össze.
\end{abstract}

% ---------------------------- T A R T A L O M J E G Y Z É K ----------------------------
\newpage \vspace*{1cm}
\thispagestyle{plain}                                                                                                                                             
%\pagenumbering{roman} \setcounter{page}{1}
\tableofcontents

\vspace*{1cm}
\listoffigures

\vspace*{1cm}
%\listoftables

\newpage
\pagestyle{fancy}
\pagenumbering{arabic} \setcounter{page}{1}
\vspace*{1cm}

% ---------------------------- M A G A   A   D O L G O Z A T ----------------------------

%%%%%%%%%%%%%%%%%%%%%%%%%%%%%%%%%%%%%%%%%%%%%%%%%%%%%%%%%%%%%%%%%%%

\section{Bevezetés}
\label{sec:bev}

A Föld-Hold rendszerének mozgását vizsgáltam síkban. A feladat során geocentrikus koordinátákkal dolgoztam. A mozgást Newton általános mozgástörvénye írja le:
\begin{center}
\begin{equation}
\label{newtequ}
F=\gamma\frac{m\cdot M}{r^2}
\end{equation}
\end{center}
Az Euler módszer során a Földet rögzítettem az origóban (0;0), a negyedrendű Runge-Kutta módszer során a Föld és a Hold is szabadon mozoghatott, a megadott kezdőfeltételeknek megfelelően. A feladat tulajdonképpen differenciálegyenletek numerikus integrálása, különböző pontosságú módszerekkel.



%%%%%%%%%%%%%%%%%%%%%%%%%%%%%%%%%%%%%%%%%%%%%%%%%%%%%%%%%%%%%%%%%%%%%%%%%%%%%%%%%%%%%%%
\section{Programok}
\label{sec:prog}

\subsection{Programok írása}
\label{progiras}

A programokat C nyelven írtam, először CodeBlocks környezetben a \cite{artic:negyedik} forrás alapján, majd Visual Studio Code-ban pontosítva, alakítva a jelenlegi feladathoz. A Visual Studio Code-ban a Microsoft Visual Studio 2017 programjába épített fordító használatával teszteltem a programokat működés közben. A program írása és finomítása során Git-es verziókövetést használtam. Létrehoztam egy repositoryt GitHub-on, amit a \url{https://github.com/graczgeri/foldhold} link alatt lehet elérni.A Git használatával jól össze tudtam hasonlítani a programrészeket, verziókat, jobban követhető a munka folyamata. Több branch-et is létrehoztam az áttekinthetőség kedvéért, ezeket - miután nem volt rá szükség és illesztettem a master branchbe a megfelelő részeket - töröltem. A két módszer összehasonlításának szemléltetésére gnuplot segítségével plotoltam a kapott adatsorokat. Külön könyvtárszerkezetet hoztam létre a különböző fileoknak (illetve a jegyzőkönyv miatt is). A teljes projektet CMake-kel fogtam össze és buildeltem. A könyvtárszerkezettel együtt, külön CMakeLists file-okat készítettem, az adott könyvtárban végzett feladatokra. A főkönyvtárba így egy egyszerű, pár soros file került, ami az alkönyvtárakat és azokban végzett feladatokat fogta össze. Mivel az exe file nélkül nem tud elkészülni az adatsor, illetve anélkül nincs miből ábrát készíteni, így egymástól függővé tettem a különböző programelemeket. A CMake használata végett külön scriptet írtam az ábrák elkészítésére és egy előre gyártott tex mintát módosítottam a saját felhasználásnak megfelelően. Sajnos a pdf file elkészítése nem sikerült a CMake használatával, így azt külön csináltam meg. A CMake használata során a \cite{artic:elso,artic:masodik} forrásokra támaszkodtam.
\subsection{Euler módszer}
\label{subsec:euler_1}

Az Euler módszer lényege, hogy $\delta$x (általában h-val jelölt) diszkrét lépéshosszal léptetjük a változók értékét. Az összes változót léptetjük egyszerre:
\begin{center}
\begin{equation}
y_{n+1}=y_n+h\cdot f(x_n,y_n)
\end{equation}
\end{center}
Eközben a független változót is léptetjük:
\begin{center}
\begin{equation}
x_{n+1}=x_n+h
\end{equation}
\end{center}
A módszer hibája, hogy ha a derivált a lépés során túl gyorsan változik, akkor a lépésnek relatív nagy lesz a hibája. Sok lépés után nagy lesz az eltérés az analitikus megoldástól, mivel a hiba a lépések számával felösszegződik.

\subsection{Runge-Kutta 4 módszer}
\label{subsec:rk4_1}

Az Euler módszer javításának tekinthető a középponti módszer. Maga az Euler módszer aszimmetrikus, mivel a deriváltat mindig az $x_n$ helyen vesszük. A középponti módszer lényege, hogy teszünk egy fél lépést, kiszámoljuk a deriváltat a középső pontban és ezt használjuk a teljes lépés során. A Runge-Kutta módszer ezt használja fel, több lépésben:
\begin{center}
\begin{equation}
k_1=h\cdot f(x_n,y_n)\newline
\end{equation}
\begin{equation}
k_2=h\cdot f(x_n+\frac{1}{2}h,y_n+\frac{1}{2}k_1)\newline
\end{equation}
\begin{equation}
k_3=h\cdot f(x_n+\frac{1}{2}h,y_n+\frac{1}{2}k_2)\newline
\end{equation}
\begin{equation}
k_4=h\cdot f(x_n+h,y_n+k_3)\newline
\end{equation}
\begin{equation}
y_{n+1}=y_n+\frac{1}{6}k_1+\frac{1}{3}k_2+\frac{1}{3}k_3+\frac{1}{6}k_4+O(h^5)
\end{equation}
\end{center}
Amiben $k_{(1..4)}$ a lépések, $O(h^5)$ pedig a lecsökkent hiba az analitikus eredménytől. A lépések együtthatóit a Butcher-tábla adja.

\section{Eredmény}
\label{sec:ered}

\subsection{Euler módszer}
\label{sebsec:euler_2}

A kiértékelés során a program az euler.dat adattömbbe írta a Hold x és y koordinátáit. A kapott adatokat ábrázolva látható az \ref{fig:euler}. ábrán a Hold Föld körüli pályája, rögzített Föld és Euler közelítés esetén.

\begin{figure}[h!]
\begin{center}
\includegraphics[width=0.9\textwidth]{@EULER@}
  \caption{\textit{Euler módszer}}
\label{fig:euler}
\end{center}
\end{figure}

\subsection{Runge-Kutta 4 módszer}
\label{subsec:rk4_2}
A kiértékelés során a program egy tömbbe írta a számított adatokat, majd azokat kiírta az rk4.dat adatfileba. A tömb egyszerre tartalmazta az időt, a Föld és Hold x és y koordinátáit, illetve az x és y irányú sebességkomponenseit. A Hold koordinátáit ábrázolva látható a \ref{fig:rk4}. ábrán, hogy sokkal szebben záródik a pálya, mint az Euler módszernél.

\begin{figure}[h!]
\begin{center}
\includegraphics[width=0.9\textwidth]{@RK4@}
  \caption{\textit{Runge-Kutta 4 módszer}}
\label{fig:rk4}
\end{center}
\end{figure}

%%%%%%%%%%%%%%%%%%%%%%%%%%%%%%%%%%%%%%%%%%%%%%%%%%%%%%%%%%%%%%%%%%%%%%%%%%%%%%%%%%%%%%%%%%
% ez ugyanazt tudja, mint a \newpage, azzal a különbséggel, hogy hogyha vannak még függő 
% ábrák vagy táblázatok, amik még nem lettek elhelyezve, akkor az most itt megtörténik, és 
% csak utána megyünk tovább. A \newpage most nekünk pl hatástalan lenne.

\cleardoublepage
\vspace*{2cm}

% Mivel *-al jelöljük meg az új section-t, nem fog bekerülni a tartalomjegyzékbe, hacsak kézzel hozzá nem adjuk
\addcontentsline{toc}{section}{Köszönetnyilvánítás}
\section*{Köszönetnyilvánítás}
Szeretném megköszönni a segítséget a gyakorlatvezetőknek, Bíró Gábornak és Nagy-Egri Máténak!

%%%%%%%%%%%%%%%%%%%%%%%%%%%%%%%%%%%%%%%%%%%%%%%%%%%%%%%%%%%%%%%%%%%%%%%%%%%%%%%%%%%%%%%%%%5

\newpage \vspace*{2cm}

\addcontentsline{toc}{section}{Irodalomjegyzék}

\bibliographystyle{src/abeld}       
% van egy csomó féle/fajta stílus, google segít bennük. Én ezt használtam a diplomamunkámhoz. 
% A helyes működéshez az kell, hogy az abeld.bst ugyanebben a mappában legyen.
\bibliography{src/references}       
% a references.bib tartalmazza egyenként a hivatkozásokat, így a tex fájlod tisztább maradhat.
%
% 
%%%%%%%%%%%%%%%%%%%%%%%%%%%%%%%%%%%%%%%%%%%%%%%%%%%%%%%%%%%%%%%%%%%%%%%%%%%%%%%%%%%%%%%%

% ----------------------------------- A P P E N D I X -----------------------------------

%\newpage \vspace*{2cm}

%\pagestyle{fancy}

%\appendix
%\section{Ez egy függelék}
%\label{app:fugg}

%És igen, ezeket is lehet \ref{app:fugg}. módon hivatkozni.

% -------------------------------- N Y I L A T K O Z A T --------------------------------

% Nagyon hosszú dokumentumoknál (pl könyv) érdemes több részre bontani a forráskódot is.
% Az \input parancs után következő fájl egyszerűen "bemásolódik", így a fordító egyben 
% fogja látni az egészet.
\newpage
\input src/nyilatkozat.tex  % EZT NE HAGYD KI! OTT KELL LENNIE A DOLGOZAT VÉGÉN!


\end{document}

Ez pedig már nem fog megjelenni, mivel az end document után vagyunk.
